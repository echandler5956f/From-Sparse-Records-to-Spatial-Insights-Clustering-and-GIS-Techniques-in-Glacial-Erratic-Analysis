\chapter{Methodology}
\label{chapter:method}

\section{Interdisciplinary Platform Development Methodology}
\label{sec:methodology_overview}

The digital heritage platform developed in this research demonstrates how interdisciplinary design methodologies can bridge geological science and cultural preservation through accessible public interfaces. This approach integrates user experience design principles with digital humanities best practices to create a web-based platform that serves multiple communities: heritage tourists seeking engaging site information, educators exploring geological and cultural connections, and researchers investigating spatial patterns in landscape significance \cite{Gregory2013, Bodenhamer2010}. The platform development methodology prioritizes public accessibility and heritage tourism functionality while maintaining scholarly integrity, creating an inclusive digital space where diverse audiences can explore the intersection of geological processes and cultural meaning-making. Central to this approach are the design insights derived from the preceding case studies (Section \ref{chapter:cases}), which revealed how the diversity of North American erratics requires sophisticated curatorial decisions about representation, accessibility, and cultural sensitivity in digital heritage contexts.

\section{Platform Architecture and Data Design}
\label{sec:platform_architecture}

The platform employs a full-stack web architecture designed to make glacial erratic information accessible to diverse public audiences while maintaining technical robustness for heritage tourism applications. The frontend utilizes React.js with Leaflet.js mapping libraries to provide responsive, interactive mapping experiences across desktop and mobile devices. The backend implements Node.js with Express.js for API management and real-time route optimization capabilities, while PostgreSQL with PostGIS extensions manages spatial data operations and geographic calculations \cite{HofmannWellenhof2006}. Each erratic is represented as a point feature using World Geodetic System 1984 (WGS84, EPSG:4326) coordinates, ensuring compatibility with standard GPS devices and web mapping services that heritage tourists utilize for navigation. Distance calculations for route optimization employ the Haversine formula to account for Earth's curvature across continental scales \cite{Snyder1987}. The database schema stores comprehensive information including geological characteristics (rock type, estimated size, transport distance), historical context (discovery accounts, cultural significance), and descriptive textual content, while the API and frontend present this information through intuitive filtering and search interfaces designed for general public use.

The platform integrates authoritative external geospatial datasets to provide contextual information that enhances both heritage exploration and educational understanding of erratics within their broader landscape settings \cite{Worboys2004}. These layers support multiple user functionalities, from TSP route optimization calculations to historical context visualization:
\begin{itemize}
    \item \textbf{Hydrological Networks:} Vector data representing rivers/streams (polylines) and lakes/water bodies (polygons) from sources like HydroSHEDS, enabling proximity calculations to nearest water features and supporting interpretation of historical accessibility patterns and settlement relationships.
    \item \textbf{Settlement Data:} Point locations of modern population centers from OpenStreetMap and historical settlement sites from sources like NHGIS, providing accessibility context for heritage tourism planning and historical interpretation of how erratics functioned as community landmarks.
    \item \textbf{Transportation Networks:} Current road networks (OpenStreetMap polylines) integrated with the TSP route optimization algorithm, plus North American road datasets from the US Department of Transportation for understanding erratics' accessibility and roles as navigation references across different time periods.
    \item \textbf{Indigenous Territories:} Polygon datasets representing historical and contemporary territorial boundaries, treaties, and linguistic regions, providing essential cultural context and ensuring respectful acknowledgment of Indigenous connections to these landscape features.
\end{itemize}
The platform implements efficient spatial indexing using PostGIS R-tree structures \cite{Guttman1984} to enable real-time proximity calculations, filtering operations, and TSP route optimization that provide responsive user experiences essential for effective web-based heritage tourism applications. Data preprocessing optimizes vector layers for web delivery while maintaining spatial accuracy needed for route calculations and proximity analysis.

\section{Interactive Mapping and Route Optimization}
\label{sec:interactive_mapping}

\subsection{Distance Calculations for Heritage Tourism}
\label{subsec:distance_calculations}

The platform's distance calculations serve two primary user-oriented functions: providing contextual information about erratics' landscape relationships and enabling efficient route optimization for heritage tourism. Due to the continental scale of the dataset, all distance calculations account for Earth's curvature using geodesic methods. The Haversine formula provides an accurate and computationally efficient approach for calculating great-circle distances between geographic points $(\text{lat}_1, \text{lon}_1)$ and $(\text{lat}_2, \text{lon}_2)$ on Earth's surface \cite{Sinnott1984}:
\begin{align*}
    a &= \sin^2\left(\frac{\Delta \text{lat}}{2}\right) + \cos(\text{lat}_1) \cos(\text{lat}_2) \sin^2\left(\frac{\Delta \text{lon}}{2}\right) \\
    c &= 2 \cdot \text{atan2}(\sqrt{a}, \sqrt{1-a}) \\
    d &= R \cdot c
\end{align*}
where $\Delta \text{lat} = \text{lat}_2 - \text{lat}_1$, $\Delta \text{lon} = \text{lon}_2 - \text{lon}_1$, and $R$ represents Earth's radius.

The platform calculates proximity relationships between erratics and contextual features (rivers, settlements, roads) to provide users with meaningful spatial context about accessibility, historical significance, and landscape setting. These calculations utilize PostGIS spatial functions including `ST\_Distance` and `ST\_ClosestPoint` \cite{Goodchild1992}, with efficient spatial indexing (R-tree algorithm) \cite{Guttman1984} ensuring responsive performance for interactive mapping applications. Distance information enhances user understanding of how erratics relate to historical travel routes, water sources, and community centers, supporting both educational exploration and heritage tourism planning.

\subsection{Traveling Salesman Problem Route Optimization}
\label{subsec:tsp_optimization}

A key innovation of the platform is real-time route optimization using Traveling Salesman Problem (TSP) algorithms, enabling heritage tourists to generate efficient visiting routes for filtered sets of erratics. The TSP implementation employs a nearest-neighbor construction heuristic followed by 2-opt local search improvement, providing practical route solutions for typical user scenarios involving 10-50 erratics \cite{Burrough2015}. The algorithm constructs distance matrices using the Haversine formula for geographic accuracy, then builds initial routes by iteratively selecting the nearest unvisited erratic from the current position. The 2-opt improvement phase systematically evaluates edge swaps to eliminate route crossings and reduce total travel distance.

The route optimization integrates user location when available through browser geolocation APIs, positioning tourists' current coordinates as the starting point for optimized touring routes. Computational complexity remains manageable for interactive applications: the nearest-neighbor construction operates in $O(n^2)$ time, while 2-opt improvement provides significant route quality enhancement with reasonable computational overhead for real-time web applications. This functionality transforms static heritage information into dynamic tourism tools, enabling users to discover multiple erratics efficiently while accommodating practical constraints like travel time and geographic preferences expressed through the platform's filtering system.

\subsection{Interactive Filtering and Search Capabilities}
\label{subsec:filtering_search}

The platform implements comprehensive filtering and search functionality that enables users to explore erratics based on diverse criteria that reflect both geological and cultural characteristics. Users can filter by geological attributes (rock type, size categories, estimated transport distance), cultural significance indicators (Indigenous heritage sites, historical landmarks, inscribed boulders), geographic context (proximity to water bodies, settlements, protected areas), and accessibility factors (road access, trail availability, land ownership status). The filtering system operates in real-time, dynamically updating both the map display and available route optimization options as users refine their selection criteria.

Search capabilities include both text-based queries across descriptive content and spatial searches within user-defined geographic boundaries. The platform's database indexing ensures responsive performance even when applying multiple filter criteria simultaneously to the full North American dataset. This functionality transforms the platform from a static information repository into an interactive exploration tool that supports diverse user goals: heritage tourists planning multi-site visits, educators designing field trips, researchers investigating regional patterns, and local communities documenting cultural landscapes. The integration of filtering with route optimization enables users to discover previously unknown erratics that align with their interests while generating practical visiting routes.

\section{Content Management and User Experience Design}
\label{sec:content_ux}

\subsection{Curatorial Approach to Content Integration}
\label{subsec:content_integration}

The platform adopts a systematic curatorial approach to integrating diverse textual materials about each erratic into accessible digital presentations. Content sources include official geological descriptions, historical accounts, Indigenous cultural knowledge where available and appropriate, local folklore and traditions, tourism materials, and academic research findings. Rather than treating this heterogeneous information as a problem to be algorithmically resolved, the platform embraces the multiplicity of perspectives as part of each erratic's cultural significance, creating structured presentations that allow users to explore different knowledge systems and interpretive frameworks while maintaining clear attribution and cultural sensitivity.

\subsection{Information Architecture and Content Presentation}
\label{subsec:information_architecture}

The platform employs structured information architecture principles to organize diverse content types into coherent, accessible presentations for public audiences. Each erratic's content is organized into standardized sections including geological characteristics, historical context, cultural significance, accessibility information, and visitor guidance \cite{Manning2008, Jurafsky2009}. This structured approach ensures consistency across the dataset while accommodating the diverse nature of available information about different erratics.

Content presentation strategies prioritize clarity and accessibility over exhaustive detail, recognizing that heritage tourism and educational users benefit from well-curated information rather than comprehensive but overwhelming documentation. The platform implements responsive design principles that adapt content display across desktop and mobile devices, ensuring effective user experiences for field-based heritage tourism applications. Interactive elements including expandable sections, multimedia integration capabilities, and contextual linking between related erratics enhance user engagement while maintaining focus on essential information that supports heritage exploration and learning objectives.

\subsection{Dynamic Filtering System Design}
\label{subsec:filtering_system}

The platform implements a sophisticated real-time filtering system that enables users to explore erratics through multiple criteria that reflect both geological and cultural characteristics. The filtering interface allows users to add, edit, and manage multiple simultaneous filters across diverse data attributes including size ranges, proximity to landscape features (water bodies, settlements, roads, trails, native territories), geological properties (rock type, elevation, displacement distance), cultural indicators (usage types, inscriptions, cultural significance scores), and accessibility metrics \cite{Reimers2019}. The system provides immediate visual feedback as users adjust filter parameters, with the map display and route optimization options updating dynamically to reflect the current selection.

The filtering architecture supports diverse exploration approaches through flexible filter types: range filters for continuous variables (size, elevation, proximity distances), categorical selections for discrete attributes (rock types, terrain characteristics, usage types), boolean toggles for binary properties (presence of inscriptions), and multi-criteria combinations that accommodate complex user requirements. This comprehensive filtering capability transforms static heritage information into an interactive exploration tool that serves heritage tourists planning visits, educators designing curricula, researchers investigating patterns, and local communities documenting cultural landscapes. The integration of filtering with TSP route optimization ensures that users can efficiently discover and visit erratics that match their specific interests and accessibility requirements.

\subsection{Thematic Organization and Content Categorization}
\label{subsec:thematic_organization}

The platform employs systematic thematic organization strategies based on curatorial expertise and user needs analysis rather than algorithmic topic discovery. Content categorization reflects the multifaceted nature of erratics as both geological specimens and cultural landmarks, organizing information around themes that facilitate user exploration and comparative understanding \cite{Grootendorst2022}. Primary thematic categories include geological features (rock composition, glacial transport evidence, regional patterns), cultural significance (Indigenous sacred sites, colonial historical markers, community symbols), human interactions (inscriptions and modifications, preservation efforts, tourism development), and landscape context (accessibility, environmental setting, related features).

This human-curated approach to content organization ensures that thematic categories reflect meaningful distinctions for heritage tourists, educators, and researchers while maintaining sensitivity to cultural protocols around Indigenous knowledge and sacred sites \cite{McInnes2018UMAP, Campello2013HDBSCAN}. The categorization system operates flexibly, allowing individual erratics to belong to multiple themes and enabling users to discover connections between geological processes and cultural meanings. Rather than imposing rigid taxonomic boundaries, the platform's thematic structure supports exploratory learning and cross-cultural understanding, helping users appreciate how different communities have engaged with these remarkable landscape features across time \cite{Blei2003, Roder2015}.

\subsection{User Engagement and Educational Framework}
\label{subsec:user_engagement}

The platform's user engagement strategies draw from digital humanities and public history methodologies to create meaningful connections between users and erratics across diverse knowledge backgrounds and cultural perspectives. Educational frameworks integrate geological science communication with cultural heritage interpretation, helping users understand how ice age processes created the physical foundations for subsequent human interactions and meaning-making. Interactive features encourage active exploration rather than passive consumption, with comparative tools that allow users to examine similarities and differences between erratics across regions, size categories, and cultural significance levels.

The platform accommodates diverse user goals through flexible engagement pathways: casual heritage tourists can access essential information and route planning without overwhelming technical detail, while educators and researchers can access more comprehensive geological and historical context. Community documentation features enable local stakeholders to contribute additional knowledge about erratics' cultural roles and contemporary significance, fostering collaborative heritage preservation that honors both scientific understanding and community connections to these remarkable landscape features. This approach recognizes that effective public engagement with geological heritage requires bridging multiple knowledge systems and cultural perspectives rather than privileging any single interpretive framework.

\section{Platform Design Integration}
\label{sec:integration}

The digital heritage platform achieves coherence through systematic integration of its core components: technical architecture, content management, filtering capabilities, and user experience design. The React.js frontend coordinates seamlessly with the Node.js/PostgreSQL backend to provide responsive user interactions, while the PostGIS spatial database enables real-time proximity calculations that support both filtering operations and TSP route optimization. Content presentation draws from the structured information architecture to populate map popups, detail panels, and sleek, styled visuals, ensuring consistent user experiences across different exploration pathways. The filtering system interfaces directly with both the spatial database and the mapping interface, enabling users to refine their exploration focus while maintaining visual clarity about available options and route possibilities.

This integrated design approach prioritizes user workflow coherence over technical complexity, ensuring that geological data, cultural narratives, accessibility information, and route planning capabilities combine into intuitive heritage exploration experiences. The platform's modularity allows different user communities—heritage tourists, educators, researchers, local stakeholders—to access the same underlying data through interfaces optimized for their particular goals, while the shared technical foundation ensures consistency and reliability. By treating digital heritage as a design problem rather than a technical challenge, the platform demonstrates how thoughtful integration of spatial technology, content curation, and user experience principles can bridge scientific and cultural approaches to landscape exploration.

\section{Digital Heritage Design Principles}
\label{sec:design_principles}

As revealed through the preceding Case Studies (Section \ref{chapter:cases}), the development of digital heritage platforms for named glacial erratics requires thoughtful design principles that acknowledge complexity while prioritizing public accessibility and cultural sensitivity \cite{Gregory2013}. Our platform design incorporates key principles that guide decision-making across technical implementation and user experience:

\begin{itemize}
    \item \textbf{Inclusive Representation Across Uneven Archives:} The platform accommodates varying levels of documentation quality and completeness through flexible content structures that highlight available information while clearly acknowledging gaps or uncertainties. Rather than excluding erratics with sparse documentation, the design celebrates diverse archival landscapes and encourages future research and community contributions to expand knowledge.
    \item \textbf{Temporal Sensitivity and Movement Narratives:} The platform addresses historical relocation and change through storytelling approaches that integrate movement into cultural heritage narratives rather than treating it as a technical problem. Cases like Plymouth Rock and Dighton Rock demonstrate how relocation becomes part of their historical significance, acknowledged through content presentation that centers current accessibility while incorporating movement stories.
    \item \textbf{Flexible Representation for Complex Features:} The design accommodates erratics that challenge simple definitions (collections like Babson's Boulders, composite formations like Judges Cave) through adaptive information architecture that presents complex spatial relationships clearly without forcing inappropriate categorical boundaries. This flexibility honors the diversity of erratic types while maintaining usable public interfaces.
    \item \textbf{Multifaceted Identity Acknowledgment:} The platform recognizes that erratics simultaneously embody geological, historical, cultural, and spiritual significance through information organization that allows multiple perspectives to coexist. Features like multi-label categorization and flexible thematic organization enable users to explore erratics from different knowledge frameworks (as essential for sites like the Willamette Meteorite/\emph{Tomanowos}).
    \item \textbf{Scale-Responsive Design and Wonder Communication:} The platform addresses extreme size variations (from modest stones to monumental boulders like Okotoks and Madison) through design strategies that convey geological wonder and help users appreciate scale relationships. Rather than viewing outliers as problems, exceptional erratics become opportunities to communicate the power of glacial processes and landscape transformation.
    \item \textbf{Cultural Protocol and Indigenous Knowledge Integration:} The design prioritizes respectful representation of Indigenous cultural knowledge and sacred sites through collaborative approaches that honor different knowledge systems. This includes prominent acknowledgment of Indigenous names and narratives (like \emph{Tomanowos}, Napi stories), careful attention to cultural protocols, and recognition that some knowledge may not be appropriate for public digital platforms.
\end{itemize}
These design principles demonstrate how digital heritage platforms can acknowledge complexity and uncertainty while serving diverse public communities. Rather than pursuing technical solutions to cultural questions, the approach embraces the rich, multifaceted nature of erratics as landscape features that bridge geological science and human meaning-making.
