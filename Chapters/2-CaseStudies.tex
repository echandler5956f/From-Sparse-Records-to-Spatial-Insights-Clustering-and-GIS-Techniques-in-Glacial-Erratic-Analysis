\chapter{Case Studies}
\label{chapter:cases}

\section{Representative Erratics: Diversity and Digital Heritage Design Decisions}
\label{sec:diversity}

Named glacial erratics across North America represent compelling subjects for digital heritage preservation, embodying a complex intersection of geological processes, historical narratives, and cultural meanings. These lithic travelers, carried vast distances by Pleistocene ice sheets and deposited often incongruously upon the landscape, function simultaneously as scientific evidence of past glacial dynamics and as focal points for human interaction, belief systems, and historical narratives \cite{Flint1971, Benn2010}. The challenge for digital heritage platforms lies not in capturing every nuance of their complex histories, but in creating accessible, standardized representations that serve both educational and research purposes while acknowledging the rich diversity of their individual characteristics.

Our digital heritage platform employs a standardized data model and user interface designed to make information about named erratics accessible to diverse audiences, from researchers to heritage tourists (as detailed in Section \ref{chapter:method}). The platform's development was informed by careful examination of representative erratics that illustrate the spectrum of geological, historical, and cultural complexity present across North America. These examples revealed the fundamental design challenge facing digital heritage projects: balancing the rich, multifaceted histories of individual cultural objects with the need for consistent, searchable, and accessible digital representations. Factors such as historical movement and fragmentation, contested cultural interpretations, extreme variations in physical scale, questions of individual versus collective representation, and varying levels of documented cultural significance all influenced our approach to data modeling and user interface design.

This section examines nine prominent North American glacial erratics—Plymouth Rock, Dighton Rock, Okotoks "Big Rock" Erratic, Willamette Meteorite, Babson's Boulders, Madison Boulder, Bleasdell Boulder, Rollstone Boulder, and Judges Cave—selected for their individual significance and because they collectively demonstrate the diversity of characteristics present across the dataset. By examining the historical, cultural, and physical particularities of each case, we illustrate both the richness of individual erratic histories and the rationale for adopting a standardized digital heritage approach. Rather than attempting to capture every nuance of their complex histories, our platform employs consistent data structures and representation methods that enable comparative analysis, public accessibility, and heritage tourism applications. These case studies serve as the foundation for understanding our methodological approach detailed in Section \ref{chapter:method}, demonstrating how the diversity of North American erratics informed our design decisions about balancing historical complexity with digital accessibility. Understanding these examples is fundamental to appreciating the digital heritage framework we developed—one that prioritizes broad public engagement and educational utility while acknowledging the individual richness that extends beyond what any single platform can fully represent \cite{Cuffey2010, Delcourt1991}.

\section{Individual Cases}
\label{sec:individual_cases}

\subsection{Plymouth Rock (Plymouth, Massachusetts): A Symbol Moved, Fractured, and Reified}
\label{subsec:plymouth}

Perhaps no glacial erratic in North America carries the symbolic weight of Plymouth Rock, enshrined in national mythology as the disembarkation point of the Mayflower Pilgrims in 1620 \cite{Seelye1997}. Yet, from a geo-analytical perspective, it represents a profound case of positional ambiguity, physical alteration, and the dominance of cultural narrative over geological fact, posing significant challenges for standardized data capture and analysis.

\textbf{Historical Context and Movement:} The historical record linking the Pilgrims directly to this specific rock is tenuous, emerging decades after the landing, primarily through the testimony of Elder Thomas Faunce in 1741 \cite{Seelye1997}. The rock, originally situated at the waterline of Plymouth Harbor, became an object of veneration and, consequently, modification. In 1774, during revolutionary fervor, an attempt to move the boulder resulted in its accidental splitting. The upper portion was relocated to the Town Square, later moved again near Pilgrim Hall, and finally rejoined with the base portion (which itself had been moved slightly for wharf construction) under the present-day granite portico in 1920 \cite{Seelye1997}. This complex history means Plymouth Rock has occupied at least four distinct locations, and the currently displayed artifact is only a fragment of the original Dedham granodiorite boulder \cite{Emerson1917}.

\textbf{Physical Characteristics and Alteration:} Estimates suggest the original boulder may have weighed upwards of 20 tons, whereas the current fragment is considerably smaller \cite{Emerson1917}. Decades of souvenir hunting further reduced its size before protective measures were enacted. Its geological origin is traced to exposures likely miles away, consistent with glacial transport by the Laurentide Ice Sheet \cite{Emerson1917}. However, its primary identity is now cultural, not geological.

\textbf{Digital Heritage Representation:} Our platform's treatment of Plymouth Rock demonstrates sophisticated curatorial decision-making in presenting contested historical narratives to public audiences. Rather than attempting to resolve debates about its authentic location or original size, we center the current memorial site while weaving the story of movement and fragmentation into accessible descriptive content. This curatorial approach acknowledges the rock's symbolic power—what matters most to visitors—while providing geological context (Dedham granodiorite composition, transport history) that enriches understanding. The platform enables users to discover Plymouth Rock through multiple pathways: heritage tourism, geological education, or symbolic significance, recognizing that different audiences bring different interests to this iconic American landmark.

\subsection{Dighton Rock (Berkley, Massachusetts): A Canvas of Contested Histories, Relocated}
\label{subsec:dighton}

Dighton Rock, a sandstone boulder famed for its dense and enigmatic petroglyphs, presents challenges related to contested interpretations, historical relocation, and the difficulty of classifying its primary significance. Originally located in the tidal Taunton River, its inscriptions have fueled centuries of speculation about pre-Columbian trans-oceanic contact \cite{Lenik2009, Delabarre1928}.

\textbf{Historical Context and Petroglyphs:} The \textasciitilde40-ton boulder, likely deposited by glacial action in the riverbed, bears carvings whose origins remain debated. Attributions range from Indigenous peoples (Wampanoag or earlier groups), to Portuguese explorers (Miguel Corte-Real), Norsemen, Chinese sailors, and even Phoenicians \cite{Lenik2009, Pohl1950}. Colonial accounts, like that of Rev. John Danforth in 1680 and Cotton Mather in 1712, document early European awareness and attempts at interpretation \cite{Lenik2009, Delabarre1928}. This long history of contested interpretation makes classifying its "cultural significance" complex.

\textbf{Physical Characteristics and Movement:} The rock is composed of greywacke sandstone, atypical for the immediate region, confirming its erratic nature \cite{Delabarre1928}. Crucially, in 1963, the boulder was removed from its tidal riverbed location to protect it from vandalism and erosion. It was eventually housed in a purpose-built museum within Dighton Rock State Park, located on the nearby riverbank \cite{Lenik2009, MassDCRDighton}.

\textbf{Digital Heritage Representation:} Dighton Rock exemplifies how our platform navigates interpretive controversy while maintaining scholarly integrity. Instead of privileging any single historical claim about the petroglyphs' origins, we present the rock as a site where multiple cultural narratives intersect—Indigenous heritage, colonial speculation, and archaeological investigation. The museum location becomes a strength in our representation: users can locate this now-protected cultural resource while learning about both its original riverine context and its preservation story. Our information architecture treats the competing interpretations as part of the rock's significance rather than problems to be solved, offering visitors entry points into broader conversations about cultural contact, historical evidence, and the layered meanings embedded in landscape features.

\subsection{Okotoks "Big Rock" Erratic (Okotoks, Alberta): An Outlier of Immense Scale and Sacred Significance}
\label{subsec:okotoks}

The Okotoks Erratic, known locally as "Big Rock," is a spectacular example of a glacial erratic notable for its immense size and its deep significance within Blackfoot (Siksika) traditional narratives \cite{AlbertaOkotoks, Dempsey1997}. Located on the relatively flat prairie west of Okotoks, Alberta, this quartzite behemoth presents a classic "outlier" problem for quantitative analysis and demands culturally sensitive data handling.

\textbf{Geological Context and Scale:} Part of the Foothills Erratics Train, Big Rock was transported hundreds of kilometers from the Jasper area of the Canadian Rockies by the Laurentide Ice Sheet \cite{AlbertaOkotoks, Stalker1973}. Composed of layered quartzite, it measures approximately 41 by 18 meters and stands 9 meters tall, with an estimated mass of 16,500 tonnes \cite{AlbertaOkotoks}. Its sheer size makes it one of the largest known glacial erratics globally. It has also fractured into two main pieces, likely due to frost weathering over millennia \cite{Dempsey1997}.

\textbf{Cultural Significance:} For the Blackfoot people, Big Rock is intrinsically linked to the stories of Napi (Old Man), the trickster figure and creator \cite{Dempsey1997, Klassen1995}. One prominent story explains Napi gave his robe to the rock on a hot day, later asked for it back when the weather turned cold, and was chased by the rolling rock after refusing Napi's request. Napi struck the rock with help from bats, splitting it in two \cite{Dempsey1997}. This narrative embeds the rock within a sacred geography and cosmology, marking it as far more than just a geological feature. It is designated as a Provincial Historic Resource \cite{AlbertaOkotoks}.

\textbf{Digital Heritage Representation:} Okotoks Big Rock posed unique challenges in respectful cultural representation that shaped our platform's approach to Indigenous sacred sites. We worked to ensure that Blackfoot perspectives—particularly the Napi creation stories—receive prominent presentation alongside geological information, recognizing that for many visitors, the cultural significance far outweighs the scientific data. The platform's design accommodates the rock's exceptional scale through specialized visual indicators and contextual information about the broader Foothills Erratics Train, helping users understand both its uniqueness and its place within larger glacial processes. By presenting the fragmentation story within Blackfoot oral tradition rather than purely geological terms, we demonstrate how different knowledge systems can coexist and enrich public understanding of significant landscape features.

\subsection{Willamette Meteorite (West Linn, Oregon / New York City): An Extraterrestrial Erratic with Contested Ownership and Sacred Status}
\label{subsec:willamette}

The Willamette Meteorite presents a unique classification challenge: it is undeniably extraterrestrial in origin but experienced geological transport processes akin to terrestrial erratics, likely involving both glacial ice and cataclysmic floods \cite{AMNHWillamette, Pasek2008}. Its status as a sacred object to the Clackamas Chinook people adds further layers of complexity regarding ownership, location, and interpretation.

\textbf{Origin and Transport:} This 15.5-ton iron-nickel meteorite, the largest found in North America, originated in the asteroid belt \cite{AMNHWillamette}. Its journey to Oregon's Willamette Valley remains debated but likely involved landing in western Canada or Montana during the last ice age. It was then incorporated into the Cordilleran Ice Sheet, transported southward, and likely rafted by icebergs during the Missoula Floods that scoured the region between 15,000 and 13,000 years ago, finally coming to rest near modern-day West Linn \cite{Pasek2008, Bretz1969}. This complex transport history involving both glacial ice and megafloods blurs the lines of typical glacial erratic deposition.

\textbf{Discovery and Cultural Significance:} The meteorite was known to the Clackamas Chinook people, who called it \emph{Tomanowos} ("Visitor from the Moon" or "Heavenly Visitor") and revered it for its spiritual power, using rainwater collected in its hollows for rituals \cite{statesmanjournalPiecesSacred, ourtimebdTomanowosMeteorite}. It was "rediscovered" by settler Ellis Hughes in 1902, who secretly moved it onto his own land, sparking a legal battle eventually won by the Oregon Iron and Steel Company, on whose land it had originally rested \cite{AMNHWillamette}. It was sold and eventually donated to the American Museum of Natural History (AMNH) in New York City in 1906, where it remains \cite{AMNHWillamette}. The Confederated Tribes of Grand Ronde, successors to the Clackamas, have long sought its repatriation, resulting in a landmark 2000 agreement allowing continued museum display but guaranteeing tribal access for religious and cultural purposes \cite{statesmanjournalPiecesSacred, ourtimebdTomanowosMeteorite}.

\textbf{Digital Heritage Representation:} \emph{Tomanowos} required us to expand our platform's conceptual boundaries, challenging the definition of "glacial erratic" while highlighting issues of cultural repatriation in museum contexts. Our representation strategy emphasizes the meteorite's sacred identity as \emph{Tomanowos} while providing accessible scientific context about its extraterrestrial origins and flood-related transport. The dual-location challenge—Oregon discovery site versus New York museum—becomes an opportunity to discuss how cultural objects move through different institutional and geographic contexts. We present the 2000 stewardship agreement as a positive model for cultural collaboration, helping users understand how contemporary Indigenous communities maintain connections to sacred objects regardless of their current physical location. This case demonstrates how digital platforms can honor Indigenous names and cultural protocols while serving educational and tourism functions.

\subsection{Babson's Boulders (Gloucester, Massachusetts): A Dispersed Collection Challenging Entity Definition}
\label{subsec:babson}

Located within the historically bleak and fascinating landscape of Dogtown Common on Cape Ann, Babson's Boulders represent not a single erratic but a collection of dozens of glacially deposited boulders deliberately inscribed with inspirational mottoes by philanthropist Roger Babson during the Great Depression \cite{Moore2005, wburMysteriousBoulders}. This presents a significant challenge to database schemas and spatial analysis techniques designed around individual point features.

\textbf{Historical Context:} Dogtown Common, once a thriving inland settlement in the 17th and 18th centuries, had become largely abandoned and overgrown by the 19th century, known for its poverty, isolation, and folklore involving witches and wild dogs \cite{Moore2005, Babson1940}. During the 1930s, Roger Babson, a successful economist and statistician, commissioned unemployed Finnish stonecutters to carve mottoes ("Keep Out Of Debt," "Help Mother," "Prosperity Follows Service") onto numerous large granite and gneiss erratics scattered across the common \cite{wburMysteriousBoulders}. His motives were partly philanthropic (providing work) and partly didactic, intending the inscriptions as enduring lessons for visitors \cite{Moore2005}.

\textbf{Physical Characteristics:} Dogtown is littered with thousands of glacial erratics, remnants of the Laurentide Ice Sheet's passage over the local granite bedrock. Babson selected dozens of these existing boulders, varying in size, for his project \cite{wburMysteriousBoulders}. They are not concentrated in one spot but distributed across a significant area of the common (over 3,000 acres).

\textbf{Digital Heritage Representation:} The distributed nature of Babson's Boulders influenced our platform's approach to representing collections versus individual objects. We treat Dogtown Common as a heritage landscape where geology, history, and social activism intersect, emphasizing how Roger Babson transformed existing glacial erratics into vehicles for public messaging during economic hardship. The platform guides visitors to this unique outdoor installation while explaining its dual character: remnants of Pleistocene glaciation repurposed as Depression-era folk art. Users can explore themes of unemployment relief, moral instruction, and landscape transformation, gaining appreciation for how human creativity builds upon geological foundations. This representation strategy recognizes that some heritage sites are best understood as integrated experiences rather than discrete objects.

\subsection{Madison Boulder (Madison, New Hampshire): Handling Another Giant}
\label{subsec:madison}

Madison Boulder, located in a dedicated Natural Area in New Hampshire, stands as another testament to the immense power of glacial transport and, like Okotoks Big Rock, presents challenges related to extreme scale \cite{Goldthwait1925, NHMadisonBoulder}. It is renowned as one of the largest known glacial erratics in North America.

\textbf{Geological Context and Scale:} This colossal boulder is composed of Conway granite, a distinctive local bedrock type, confirming its relatively shorter (though still significant) transport distance compared to erratics like Okotoks \cite{Goldthwait1925, NPSMadisonBoulder}. Its dimensions are roughly 25 meters long, 9 meters high, and 11 meters wide, with an estimated weight of approximately 4,662 tonnes (often cited as over 5,000 tons) \cite{NHMadisonBoulder, NPSMadisonBoulder}. It rests partially buried in a forested setting, deposited as the Laurentide Ice Sheet retreated around 14,000 years ago \cite{Goldthwait1925}.

\textbf{Historical and Cultural Context:} While lacking the deep, specific Indigenous narratives associated with Okotoks (though the region is ancestral Abenaki territory), Madison Boulder has been a known landmark and natural curiosity for centuries, attracting visitors and geological interest \cite{NHMadisonBoulder}. It was designated a National Natural Landmark in 1970 \cite{NPSMadisonBoulder}. Its significance is primarily geological and as a local point of interest.

\textbf{Digital Heritage Representation:} Madison Boulder showcases our platform's ability to convey scientific wonder through accessible presentation. The boulder's immense scale—challenging to comprehend from measurements alone—becomes tangible through comparative descriptions and contextual information about glacial transport processes. We emphasize its National Natural Landmark designation as recognition of its geological importance while explaining how Conway granite's distinctive characteristics helped scientists understand regional glacial flow patterns. The platform transforms technical geological concepts into compelling narrative about ice sheet power and landscape formation, helping visitors appreciate both the boulder's individual magnificence and its role in revealing Earth's climatic history. Varying size estimates from different sources become a teachable moment about scientific measurement challenges rather than confusion to avoid.

\subsection{Bleasdell Boulder (Trenton, Ontario): Dealing with Data Sparsity and Regional Variation}
\label{subsec:bleasdell}

The Bleasdell Boulder, located within the Bleasdell Boulder Conservation Area near Trenton, Ontario, is recognized as one of Canada's largest known glacial erratics \cite{LTCBleasdell, Chapman1984}. While a significant local landmark, its documentation illustrates the challenge of data heterogeneity and sparsity often encountered when moving beyond the most famous North American erratics.

\textbf{Geological Context and Size:} Composed of Precambrian granite gneiss, the Bleasdell Boulder measures approximately 13.4 meters long, 7.3 meters wide, and 6.7 meters high \cite{LTCBleasdell}. Its estimated weight is around 2,000 tonnes. It was transported by the Laurentide Ice Sheet from the Canadian Shield, likely hundreds of kilometers to the north, and deposited as the ice retreated \cite{Chapman1984}.

\textbf{Historical and Cultural Context:} The boulder has been a known landmark for local inhabitants, including First Nations (likely Mississauga or preceding groups) and later European settlers, for a considerable time \cite{Chapman1984}. It gained wider recognition and protection when the surrounding land was acquired for conservation in the mid-20th century \cite{LTCBleasdell}. Compared to Plymouth Rock or Dighton Rock, however, detailed historical accounts, specific Indigenous narratives, or extensive folklore directly attached to this \emph{specific} boulder appear less prevalent or less widely documented in easily accessible sources.

\textbf{Digital Heritage Representation:} Bleasdell Boulder demonstrates our commitment to inclusive representation across uneven archival landscapes. Rather than excluding sites with limited documentation, we developed presentation strategies that acknowledge gaps while highlighting available information—geological data, conservation history, and regional significance within Ontario's natural heritage system. The platform treats sparse documentation as itself historically significant, reflecting patterns of record-keeping, research attention, and cultural valuation that vary across regions and time periods. For users, Bleasdell becomes a starting point for exploring Canada's lesser-known geological landmarks and understanding how conservation efforts preserve natural features for future study. This approach ensures that spectacular erratics beyond the most famous examples gain visibility and protection through digital heritage platforms.

\subsection{Rollstone Boulder (Fitchburg, Massachusetts): Relocation and Reassembly of a Community Symbol}
\label{subsec:rollstone}

The Rollstone Boulder shares similarities with Plymouth Rock and Dighton Rock in its history of movement, but adds the complexity of fragmentation followed by deliberate reassembly \cite{googleLewistonEvening, telegramWorcesterCounty}. Originally perched atop Rollstone Hill in Fitchburg, this large porphyritic granite boulder became a cherished local symbol, only to be threatened by quarrying operations.

\textbf{Geological Context and Original Location:} The \textasciitilde110-ton boulder is composed of porphyritic granite, geologically distinct from Rollstone Hill itself (composed of gneiss), confirming its status as an erratic transported by the Laurentide Ice Sheet \cite{mindatMineralsRollstone, Hitchcock1841}. Its prominent position on the hilltop made it a natural landmark.

\textbf{Historical Context, Movement, and Reassembly:} As quarrying encroached on Rollstone Hill in the late 19th and early 20th centuries, local citizens grew concerned for the boulder's fate \cite{googleLewistonEvening, telegramWorcesterCounty}. Notably, Fitchburg naturalist James F. D. Garfield advocated for its preservation \cite{googleLewistonEvening}. In 1929-1930, a plan was enacted: the boulder was carefully blasted into manageable pieces, transported down the hill, and meticulously reassembled in a small park (Crocker Field, later relocated slightly to the Upper Common) in downtown Fitchburg, where it stands today as a monument \cite{googleLewistonEvening, telegramWorcesterCounty}.

\textbf{Digital Heritage Representation:} Rollstone Boulder's story of community rescue and reconstruction exemplifies how our platform celebrates civic heritage alongside geological significance. We foreground the extraordinary 1930 preservation effort—citizens literally moving mountains to save their landmark—as a powerful example of grassroots cultural stewardship. The platform's location data centers the current downtown site where visitors can experience the boulder today, while the narrative explains how this urban setting resulted from community action rather than natural processes. By presenting relocation and reassembly as creative solutions rather than compromises, we help users understand how communities actively shape their cultural landscapes. This case study illustrates how erratics can embody values like civic engagement and heritage preservation that extend far beyond their geological origins.

\subsection{Judges Cave (New Haven, Connecticut): A Composite Feature with Historical Significance}
\label{subsec:judges_cave}

Located within West Rock Ridge State Park near New Haven, Judges Cave is not a single glacial erratic but rather a natural shelter formed by a jumble of large trap rock boulders deposited and arranged by glacial action \cite{Dana1891, Stiles1794}. Its fame derives from its alleged use as a hiding place for two English judges who had condemned King Charles I to death.

\textbf{Geological Context:} The "cave" is a recess formed by several large basalt (trap rock) boulders, likely plucked by the Laurentide Ice Sheet from West Rock Ridge itself or nearby sources and deposited close by as the glacier retreated \cite{Dana1891, Rice1906}. It is therefore a collection of erratics creating a specific geomorphological feature, rather than a single transported stone.

\textbf{Historical Significance:} The site gained renown as the supposed hiding place, in 1661, of Edward Whalley and William Goffe, two of the 59 judges who signed the death warrant of Charles I \cite{Stiles1794, onlyinyourstateLearnFascinating}. Fleeing the Restoration government of Charles II, they sought refuge in the New England colonies. Local Puritan sympathizers reportedly sheltered them in this natural formation for several weeks \cite{Stiles1794}. A third judge, John Dixwell, also spent time in hiding in the area. The site became a point of historical pilgrimage and local lore \cite{onlyinyourstateLearnFascinating}.

\textbf{Digital Heritage Representation:} Judges Cave challenged us to represent spaces created by geological processes rather than individual objects, leading to innovative approaches for composite features. We focus on the sheltered space itself—how glacial action created a politically significant refuge—rather than cataloging constituent boulders. The platform emphasizes the cave's role in early American political history while explaining how geological chance enabled human agency: the judges' story depends entirely on ice age processes creating this particular configuration of protective stones. This integration of geological and political narrative demonstrates how natural processes and human events intersect in landscape formation. Users gain appreciation for how geological features provide the stage for historical drama, understanding both the regicide judges' desperate situation and the glacial forces that made their temporary salvation possible.

\section{Synthesis: Design Insights from Erratic Diversity}
\label{sec:synthesis}

The examination of these nine diverse case studies reveals the rich complexity that shaped our digital heritage platform's design philosophy and curatorial approach. Rather than viewing this diversity as obstacles to overcome, these examples became foundational to developing presentation strategies that honor both geological significance and cultural meaning. The erratics collectively demonstrate why digital heritage platforms must move beyond simple cataloging toward sophisticated information architecture that serves multiple audiences and purposes. Key design insights include:

\begin{enumerate}
    \item \textbf{Curatorial Decisions About Movement and Change:} Many significant erratics (Plymouth, Dighton, Rollstone, Willamette) have complex histories of relocation that required thoughtful curatorial choices about which location and context to emphasize. Rather than attempting to resolve these temporal complexities technically, we developed presentation strategies that acknowledge movement as part of cultural heritage stories. Our platform foregrounds current accessibility while incorporating movement narratives textually, recognizing that users need practical location information alongside historical understanding of how these cultural objects have traveled through time and space.
    \item \textbf{Embracing Multifaceted Identity:} The diversity of erratics—from extraterrestrial \emph{Tomanowos} \cite{AMNHWillamette, Pasek2008} to composite formations like Judges Cave \cite{Dana1891} to artistic installations like Babson's Boulders \cite{Moore2005}—demonstrated the inadequacy of rigid categorical thinking. Many erratics have evolved from natural landmarks into powerful cultural, political, or religious symbols \cite{Seelye1997, Lenik2009, googleLewistonEvening, Stiles1794}. Our platform design embraces this multifaceted character through flexible information architecture that allows users to explore erratics from multiple perspectives—geological, historical, cultural, or spiritual—without forcing singular definitions. This approach reflects digital heritage best practices that honor the complexity of cultural objects while maintaining accessibility.
    \item \textbf{Conveying Scale and Wonder:} The monumental size of erratics like Okotoks and Madison Boulder \cite{AlbertaOkotoks, NHMadisonBoulder} presented opportunities to communicate geological wonder through accessible presentation. Rather than viewing extreme scale as a technical problem, we developed user experience strategies that help visitors comprehend immense size through comparative descriptions, contextual information, and narrative about glacial transport processes. These exceptional erratics become gateways for understanding ice sheet power and geological time scales, demonstrating how outliers can enhance rather than distort public appreciation of natural phenomena \cite{Cuffey2010}.
    \item \textbf{Innovation in Spatial Representation:} Collections like Babson's Boulders and composite formations like Judges Cave challenged us to develop creative approaches for representing heritage landscapes versus discrete objects \cite{wburMysteriousBoulders, Dana1891}. Rather than forcing complex spatial relationships into simple point-location models, we designed representation strategies that honor how these sites function as integrated experiences. Some erratics are best understood as individual landmarks, others as distributed collections, and still others as functional spaces created by geological processes. This flexibility in spatial representation reflects sophisticated understanding of how different types of cultural sites serve different user needs and exploration patterns.
    \item \textbf{Inclusive Representation Across Archival Landscapes:} The dramatic variation in available documentation—from extensively studied Plymouth Rock to sparsely documented Bleasdell Boulder—required developing presentation strategies that embrace rather than exclude based on archival completeness \cite{LTCBleasdell, Delcourt1991}. We designed the platform to accommodate erratics across the full spectrum of documentation richness, recognizing that archival gaps often reflect historical patterns of research attention, cultural valuation, and resource allocation rather than intrinsic significance. This inclusive approach ensures that spectacular geological features gain visibility and protection through digital heritage platforms, regardless of their current documentation status, while acknowledging the limitations and biases inherent in historical record-keeping.
    \item \textbf{Bridging Scientific and Cultural Knowledge Systems:} The integration of geological data with cultural narratives—particularly Indigenous sacred knowledge from sites like Okotoks and \emph{Tomanowos} \cite{Delcourt1991, Klassen1995, ourtimebdTomanowosMeteorite, statesmanjournalPiecesSacred}—required developing presentation approaches that honor different ways of knowing. Rather than privileging scientific over cultural perspectives, our platform treats geological and cultural significance as complementary ways of understanding landscape meaning. We developed curatorial strategies that present Indigenous creation stories alongside glacial transport explanations, recognizing that different audiences bring different epistemological frameworks to these sites. This interdisciplinary approach demonstrates how digital heritage platforms can facilitate dialogue between scientific and humanistic knowledge systems without requiring resolution of their different truth claims.
\end{enumerate}

These design insights demonstrate that successful digital heritage platforms for complex cultural objects must move beyond simple data management toward sophisticated curatorial and user experience design. The erratics' deep entanglement with human history, culture, and perception requires presentation strategies that embrace complexity while maintaining accessibility, honor diverse knowledge systems while enabling broad public engagement, and preserve the essential character of these remarkable landscape features while translating their significance for contemporary audiences.

\section{Transition to Methodology}
\label{sec:transition}

The diverse characteristics revealed by these nine case studies—from Plymouth Rock's contested heritage narratives \cite{Seelye1997} to the extraterrestrial nature of \emph{Tomanowos} \cite{AMNHWillamette}, from the monumental scale of Okotoks \cite{AlbertaOkotoks} and Madison Boulders \cite{NHMadisonBoulder} to the complex spatial relationships of Babson's Boulders \cite{wburMysteriousBoulders} and Judges Cave \cite{Dana1891}—directly informed our platform's design philosophy and implementation approach. Rather than viewing this diversity as complications to resolve, we embraced these examples as foundations for developing digital heritage methodologies that honor complexity while serving diverse user communities.

The following section details the technical architecture and curatorial approaches of our interdisciplinary platform. It explains how we translated the design insights revealed by these case studies into practical implementation decisions, including: the development of flexible data structures that accommodate diverse types of cultural and geological information; the creation of user interface designs that support multiple exploration pathways and audience needs; the implementation of route optimization and filtering capabilities that enhance heritage tourism experiences; and the integration of spatial analysis capabilities that serve both public engagement and research applications, all designed to demonstrate how digital platforms can bridge scientific and cultural approaches to landscape exploration.
