\chapter{Introduction}
\label{chapter:intro}
Test change

Across the vast landscapes of North America shaped by Pleistocene glaciation, glacial erratics stand as prominent and often enigmatic features. These boulders, ranging from modest stones to colossal megaliths, were transported, sometimes hundreds of kilometers, by continental ice sheets and deposited far from their geological origins as the ice retreated \cite{Flint1971, Benn2010}. Their significance is twofold. Scientifically, they serve as direct physical evidence of past ice extent, flow directions, and erosional processes, providing invaluable data points for reconstructing Quaternary environments \cite{Cuffey2010}. Culturally, many named erratics have become deeply embedded in human history and perception, functioning as sacred sites in Indigenous traditions, navigational landmarks for travelers, boundary markers in colonial settlements, sources of local folklore, and even national symbols \cite{Seelye1997, Lenik2009, Dempsey1997}.

Despite their ubiquity and importance, conducting large-scale, systematic analyses of named glacial erratics presents significant methodological hurdles, primarily stemming from the nature of the available data. Unlike standardized geological surveys, information about named erratics is often scattered across diverse sources: historical texts, geological reports, archaeological surveys, local historical society records, oral histories, and digital databases of varying quality. Consequently, the data corpus is characterized by sparsity and heterogeneity \cite{Gregory2013}. Records may be geographically biased, with dense documentation in some regions (like New England) and sparse information in others. Data quality varies dramatically; some erratics have precise geological descriptions and coordinate locations, while others are documented through vague textual accounts or approximate locations. Furthermore, the data often mixes quantitative metrics (size, elevation) with qualitative, unstructured text (historical narratives, cultural significance), and must account for temporal changes, as many significant erratics have been moved, fractured, or re-contextualized over time (Section \ref{chapter:cases}).

Addressing this data challenge requires computational methods capable of integrating and analyzing these disparate, multi-modal data sources effectively. There is a need for a robust framework that can handle spatial uncertainty, synthesize quantitative spatial relationships with qualitative textual information, and identify meaningful patterns despite data sparsity and noise. This paper aims to address this gap by pursuing three primary objectives: (1) To develop and rigorously detail an integrated computational methodology combining geospatial analysis and Natural Language Processing (NLP) specifically tailored for the study of North American named glacial erratics. (2) To demonstrate, through theoretical exposition and practical considerations, how this methodology directly confronts key challenges inherent in the domain, including data sparsity, locational ambiguity, classification complexity involving both geological and cultural facets, and the representation of non-standard features like boulder fields or scale outliers. (3) To illustrate the rationale and necessity of the methodology's specific components through in-depth case studies of nine particularly complex and informative erratic examples that drove its design.

The methodology presented herein (detailed in Section \ref{chapter:method}) leverages a Geographic Information System (GIS) framework built upon a spatially-enabled database (conceptually PostgreSQL/PostGIS). This foundation supports the integration of erratic data with relevant contextual geographic layers (hydrology, settlements, transport networks, Indigenous territories). Spatial analysis techniques are employed to compute geodesic proximity metrics (using the Haversine formula \cite{Sinnott1984}) and to identify spatial patterns through density-based clustering (DBSCAN \cite{Ester1996}). Crucially, these geospatial methods are coupled with advanced NLP techniques applied to the textual corpus associated with each erratic. We utilize pre-trained Sentence Transformer models \cite{Reimers2019} to generate semantic vector embeddings, capturing nuanced meaning beyond keywords. These embeddings facilitate unsupervised topic discovery using approaches conceptually grounded in BERTopic \cite{Grootendorst2022} (embedding clustering) and Latent Dirichlet Allocation (LDA \cite{Blei2003}) (probabilistic topic modeling), allowing for the identification and classification of latent thematic content within the descriptions.

The development of this integrated pipeline was not purely theoretical but was substantially informed by confronting the complexities of real-world examples. Section \ref{chapter:cases} presents detailed case studies of nine prominent North American erratics (including Plymouth Rock, Okotoks "Big Rock", the Willamette Meteorite, Babson's Boulders, and others). These cases are presented not merely as applications, but as critical methodological drivers. Their unique characteristics—histories of movement and fragmentation, extreme scale, contested classifications, collective versus individual nature, and deep, specific cultural meanings—exposed the limitations of simpler analytical approaches and necessitated the development of the more sophisticated and flexible methods detailed in Section \ref{chapter:method}.

This paper proceeds as follows: Section \ref{chapter:cases} provides the in-depth case studies that highlight the methodological challenges. Section \ref{chapter:method} details the theoretical foundations and algorithmic components of the integrated geospatial and NLP analysis pipeline designed to address these challenges. Section \ref{chapter:results} presents the outcomes of applying this methodology to the broader dataset, discussing the emergent spatial and thematic patterns. Finally, Section \ref{chapter:conclusion} summarizes the contributions and suggest avenues for future research in the computational analysis of complex landscape features like glacial erratics.
