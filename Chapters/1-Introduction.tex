\chapter{Introduction}
\label{chapter:intro}

Across the vast landscapes of North America shaped by Pleistocene glaciation, glacial erratics represent a compelling intersection of geological processes and cultural meaning-making. These boulders, ranging from modest stones to colossal megaliths, were transported sometimes hundreds of kilometers by continental ice sheets and deposited far from their geological origins as the ice retreated \cite{Flint1971, Benn2010}. While scientifically valuable as direct physical evidence of past ice extent, flow directions, and erosional processes that provide crucial data for reconstructing Quaternary environments \cite{Cuffey2010}, many named erratics have simultaneously acquired profound cultural significance. They function as sacred sites in Indigenous traditions, navigational landmarks for travelers, boundary markers in colonial settlements, sources of local folklore, and even national symbols \cite{Seelye1997, Lenik2009, Dempsey1997}. This dual identity—as both geological specimens and cultural landmarks—positions erratics as boundary objects that exist at the intersection of scientific and humanistic inquiry.

Despite their cultural and scientific importance, systematic exploration and preservation of named glacial erratics knowledge has been hindered by the scattered nature of information across diverse sources: historical texts, geological reports, archaeological surveys, local historical society records, oral histories, and digital databases of varying quality. This fragmentation creates significant barriers not only for researchers and educators but also for public engagement with these culturally significant landscape features \cite{Gregory2013}. The challenges are multifaceted: records are often geographically biased, with dense documentation in some regions (like New England) and sparse information in others; data quality varies dramatically, with some erratics having precise geological descriptions and coordinate locations while others exist only in vague textual accounts or approximate locations; and the information frequently mixes quantitative scientific metrics with qualitative cultural narratives, requiring interdisciplinary approaches to meaningful integration. Additionally, many significant erratics have been moved, fractured, or re-contextualized over time, complicating both geological analysis and cultural heritage preservation efforts (Section \ref{chapter:cases}).

This paper presents the development of an interdisciplinary digital platform that bridges geological science and cultural heritage preservation through the creation of a publicly accessible web-based mapping interface for North American named glacial erratics. Drawing on methodologies from both digital humanities and spatial technology, the platform demonstrates how computational tools can serve dual purposes: advancing scientific accessibility while preserving and presenting cultural knowledge embedded in the landscape. The project pursues three integrated objectives: (1) To develop a comprehensive digital heritage platform that consolidates scattered information about named glacial erratics into a unified, publicly accessible interface supporting both scientific inquiry and cultural exploration through advanced filtering and route optimization capabilities. (2) To demonstrate how interdisciplinary computational methods—combining geospatial analysis, database design, and user experience principles—can enhance public engagement with complex landscape features while addressing fundamental challenges of representing cultural objects in standardized digital systems. (3) To create a robust technical and conceptual framework that serves as infrastructure for future collaborative research between geological scientists, historians, and cultural communities, enabling diverse forms of scholarly and public engagement.

The digital platform employs a full-stack architecture designed to serve both technical and public engagement goals: React.js provides an intuitive frontend interface accessible to diverse audiences, Node.js/Express enables flexible backend processing for research applications, and PostgreSQL with PostGIS extensions supports sophisticated spatial data management while maintaining performance for public web access (detailed in Section \ref{chapter:method}). Recognizing that erratics exist within broader cultural and geographical contexts, the platform integrates erratic data with relevant contextual layers including hydrology, settlements, transport networks, and Indigenous territories. A key innovation supporting heritage tourism and public engagement is the implementation of real-time Traveling Salesman Problem (TSP) route optimization, enabling users to generate practical visiting routes for culturally and scientifically significant sites based on their current location. The spatial analysis capabilities, including geodesic proximity metrics (using the Haversine formula \cite{Sinnott1984}) and spatial pattern identification through density-based clustering (DBSCAN \cite{Ester1996}), serve both scientific analysis and cultural geography exploration. The platform architecture also incorporates advanced computational linguistics techniques applied to textual descriptions, utilizing pre-trained Sentence Transformer models \cite{Reimers2019} for semantic analysis and topic modeling approaches based on BERTopic \cite{Grootendorst2022} and Latent Dirichlet Allocation (LDA \cite{Blei2003}) to support future research into cultural themes and narrative patterns within the erratic descriptions.

The platform's development was informed by careful consideration of representative examples that illustrate both the diversity and complexity of North American named erratics. Section \ref{chapter:cases} presents detailed examination of nine prominent erratics (including Plymouth Rock, Okotoks "Big Rock", the Willamette Meteorite, Babson's Boulders, and others) that demonstrate the spectrum of geological, historical, and cultural characteristics present across the dataset. These cases reveal the interdisciplinary challenges inherent in digital heritage projects: balancing historical complexity with standardized representation, accommodating diverse cultural meanings within consistent data structures, and designing public interfaces that serve both educational and research purposes. While these erratics possess rich, complex histories involving movement, fragmentation, contested interpretations, and deep cultural significance, the platform adopts a standardized approach that prioritizes accessibility and consistent analysis capabilities—a key design decision that reflects digital humanities principles of broad access over specialized complexity.

This paper proceeds as follows: Section \ref{chapter:cases} examines prominent erratics that illustrate the diversity of geological and cultural characteristics within the dataset, demonstrating the interdisciplinary design challenges addressed by the platform. Section \ref{chapter:method} details the technical architecture and implementation of the digital heritage platform, including the spatial database design, public interface development, and route optimization capabilities that serve both scientific and cultural exploration goals. Section \ref{chapter:results} presents the outcomes of the platform deployment, discussing user engagement capabilities, analytical functionality, and implications for digital heritage and public scholarship. Finally, Section \ref{chapter:conclusion} summarizes the interdisciplinary contributions and suggests directions for future development in digital platforms that bridge scientific and cultural approaches to landscape exploration.
